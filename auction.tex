%File: formatting-instruction.tex
\documentclass[letterpaper]{article}
\usepackage{ijcai15}
\usepackage{times}
\usepackage{url}
\usepackage{helvet}
\usepackage{courier}
\usepackage{graphicx}
\usepackage{subfigure}
\usepackage{slashbox}
\usepackage[lined,boxed,commentsnumbered, ruled]{algorithm2e}
\usepackage{balance}
\usepackage{amsmath}
\usepackage{xcolor}
\newtheorem{theorem}{Theorem}%[section]
\newtheorem{proposition}{Proposition}
\newenvironment{proof}{{Proof:}}{\hfill\rule{2mm}{2mm}}
\newenvironment{mechanism}{Mechanism}{\hfill\rule{2mm}{2mm}}
\newtheorem{definition}{Definition}
\newtheorem{example}{Example}
\newtheorem{lemma}{Lemma}
\newtheorem{property}{Property}

\title {Single item auctions with discrete action spaces}
\author {Someone}
 \begin{document}
% The file aaai.sty is the style file for AAAI Press
% proceedings, working notes, and technical reports.
%

\maketitle
\begin{abstract}
Most analyses in auction design have adopted two assumptions that greatly diminish their applicability. First of all, it is assumed that the seller knows the valuation distribution of the bidders. Second, the bidder can report anything within its valuation range, in particular, allowing for the truthful revelation of its valuation. We relax both assumptions by considering a setting where the seller has no knowledge on the valuation distributions and the bidders only has a discrete set of bid levels to choose from. Under this setting, we put forward a mechanism with a number of desirable properties: truthfulness (bidder always wants to bid the highest bid level below one's type), not dominated in welfare or revenue.

Given such an auction, we also propose an efficient algorithm via dynamic programming that selects a subset of bid levels that optimizes revenue. We finally test the performance of our auction on simulated data and show that its revenue is close to that of the second price auction with unconstrained (continuous) bid space.


\end{abstract}

\section{Introduction}

In the literature of auction design~\cite{Milgron2004,krishna2009auction,shoham09}, most works have been on designing {\em truthful} mechanisms. This is well justified by the {\em revelation principle}~\cite{Myerson81}, which states that it is without loss to restrict attention to truthful mechanisms. Focusing on truthful mechanisms not only reduces the design space of the seller, but also saves the efforts for counterspeculation of the buyers.

An implicit assumption in truthful mechanism design is that revelation of one's true type is always feasible. Indeed, this is not a problem in standard mechanism design setups, where it is up to the designer to determine the action spaces. However, this assumption fails to hold in many practical scenarios, where there are natural, exogenous constraints on the set of possible actions. For example, in combinatorial auctions~\cite{sandholm02algorithm} where bidders have combinatorial preferences on bundles of items, truthful revelation of such a preference requires a bidder to communicate a value on each subset of items, results in an exponential blow up in communication complexity. A practical combinatorial auction often imposes constraints on the number of package bids a bidder can place.

As another example, consider the simultaneous ascending auction that is used in the FCC spectrum allocation~\cite[Chapter 7]{Milgron2004}. To accelerate the bidding process, the auction imposes a {\em minimum bid increment}, which essentially requires that the next bid in the ascending auction is at least $10$ percent higher than the previous highest bid. Minimum increment effectively discretizes the bidding space, invalidates truthful report in most situations. Such minimum bid increment exists in eBay auctions as well. 

The above observations motivate an active line of research that concerns the {\em expressiveness} of mechanisms~\cite{Conitzer12:Computing,Parkes05:Optimize-and-Dispatch,Benisch2008,Peerapong2011}. All these works concern how to trade off between expressiveness of the bidding language and the revenue or social welfare. Auctions with discrete bid levels are also considered in \cite{naroditskiy2007using,rabinovich2013computing}, both of which concentrate on finding Bayesian-Nash Equilibria in first-price auction or second-price auction.

In this paper, we consider a practical single-item auction design setting with restricted expressiveness. In particular, the action space of all bidders is restricted to a set of discrete {\em bid levels}. With this interface, truth revelation may not be feasible and the revelation principle fails to hold. Tailored for this setting, we put forward an auction, coined the extended second price auction. Our auction resembles the second price auction when there are multiple winners tied at the highest bid; however, when there is a unique winner, our auction charges a bit more than the second price auction. We show that, our auction satisfies the following desirable properties:
\begin{itemize}
\item Truthfulness: each bidder finds in her best interests to bid the highest bid level below her true type;
\item Not dominated in welfare: there exists no other auction that can dominate our auction in welfare for every possible type profile;
\item Not dominated in revenue: there exists no other individually rational (IR) truthful auction that can dominate our auction in revenue for every possible type profile;
%\item Revenue Comparison: We show a upper bound by which our auction exceeds the second price auction in revenue, and show that this bound is tight for some instances.
\end{itemize}

We then take the investigation further by considering the following problem, given a set of possible bid levels, how to select a subset of bid levels that optimizes revenue? Similar problems have been considered in Dutch, English auctions and general mechanism design settings~\cite{rothkopf1994role,li2011revenue,sujarittanonta2010design,david2007optimal,blumrosen2013mechanism}.
\cite{mu2008mechanism} gives a sufficient and necessary condition for strategy-proof algorithms where the private information is drawn from discrete space. 
For the extended second price auction, we propose an efficient algorithm via dynamic programming that selects the optimal subset of bid levels.

We finally test the performance of the extended second price auction on simulated data and show that its revenue is close to that of the second price auction with unconstrained (continuous) bid space.


\section{The Settings}

We consider a setting that is built upon the standard independent private value (IPV) setting. In an IPV setting, there is a seller who has an indivisible item for sale. Her valuation of the item is zero. There is a set $M$ of $m$ bidders that are interested in the item. The degree of interest is expressed by a number called valuation. Each bidder $i$ knows her own valuation $v_i$. Other bidders, denoted by $-i$, treat $v_i$ as a random variable that is distributed according to some distribution function $F_i(v_i)$ that is positive everywhere on $[0, v_{max}]$. $v_i$ is independent of $v_j$ for any $i\neq j$. An (direct) auction then solicits bids from the bidders, allocates the item to the bidders according to its {allocation rule} and charges payments according to its payment rule.

Distinct from the standard IPV setting, which places no restrictions on the set of bids to report\footnote{By revelation principle, it suffices to restrict attentions to $[0, v_{max}]$ for each $i$}, our setting only allows bidders to select bids from a given list of {\em bid levels} $\mathcal{B}=\{l_1,l_2,\ldots,l_n\}$ such that $v_{max}>l_1>\ldots,l_n=0$. Distinct from the standard IPV setting, in our setting the seller does not have any knowledge about valuation distribution. Therefore, our setting effectively excludes cases such as full-information setting where the seller knows the exact value of each bidder, as well as Myerson's standard setting of optimal Bayesian auction design\footnote{In our experimental section, we do assume that the seller has some knowledge on the distribution so that she can select an optimal subset of bid levels}.

Note that, the standard concepts such as truthfulness, welfare and revenue optimality are no longer well-defined. We give several definitions that are tailored for our setting.

%to make our theoretical results general, we don't have many assumptions on distribution.
%We concern about the case where the types of the bidders are on the full-domain, $R^+$ without a specific probability density function.
%Without loss of generality, we assume the type of each bidder is above the lowest bid level.
%In this setting, efficiency is no longer well-define.
%Alternatively, we care about non-dominance.
%Intuitively, a mechanism is non-dominant if it is not dominated by another mechanism on any type profile.

%However, in our experiments, we do have some assumptions on the distribution of types.

%Question 1. Is $\mathcal{B}\in [a_i, b_i]$?
%Question 2. What is prior free?
%
%In this paper, we focus on single-item sealed-bid auction with discrete bids and continuous types.
%In the market, $m$ bidders, numbered from 1 to $m$, bid for a single item with positive bids.
%These bids form a \textit{bidding profile}.
%All the bids belong to a given set $\mathcal{B}$.
%We call the elements in $\mathcal{B}=\{l_1,l_2,\ldots,l_n\}$ \textit{bid levels}.
%The elements in $\mathcal{B}$ follow decreasing order.
%There exists an $\epsilon$, such that for any $b_i$ and $b_j$ in $\mathcal{B}$, $abs(b_i-b_j)>\epsilon$.
%The types (values) of bidders $\mathcal{V}=\{v_1,v_2,\ldots,v_n\}$ are distributed on $R^+$, which is a private information.

%In order to define the desired properties, we give the definition of truthfulness as follows, which makes it straightforward for bidders to choose their bids in a truthful mechanism.

\begin{definition}
	An auction is {\em truthful} if it is a dominant strategy for each bidder to report the highest bid level below (or equal to) her type.
	%A bidder is truthful if their bid is the closest bid level no more than their type.
\end{definition}

It is worth pointing out that, restricting attention to truthful auctions is no longer without loss of generality, since the revelation principle no longer holds. However, we argue that the above truthfulness property is still desirable in that it suggests a simple dominant strategy for each bidder to follow.

\begin{definition}
	An auction is {\em ex-post individually rational} if the utility of every bidder is no less than 0 when following an equilibrium.
\end{definition}

We will show later that, in our setting, there is no mechanism that can achieve the standard efficiency (item always goes to the highest value bidder). This motivates the definition of the following efficiency notion.

\begin{definition}
	An auction $M$ is not dominated in welfare if there does not exist an auction $\mathcal{M'}$, such that $\mathcal{M'}$ yields at least as much social welfare as $\mathcal{M}$ on any type realization and yields strictly more on some type realization.
\end{definition}

Similarly, we can define non-dominance in revenue.

\begin{definition}
	An auction $M$ is not dominated in revenue if there does not exist an auction $\mathcal{M'}$, such that $\mathcal{M'}$ yields at least as much revenue as $\mathcal{M}$ on any type realization and yields strictly more on some type realization.
\end{definition}


\section{Extended second price auction}

In this section, we present an auction that achieves all the desired properties above.
The auction is coined \textit{extended second price auction} (ESP), for its similarities to the second-price auction (SP).

\subsection{Mechanism description}

In cases where there are several highest bids tied at some bid level. The ESP is exactly the same as SP. However, when the highest bid is different from the second highest, the winner needs to pay a bit more than what they pay in SP, i.e., the second highest bid. The winner's payment is independent from their own bid, which ensures the truthfulness of the auction. The detailed description of ESP is as follows.


\begin{algorithm}
	\caption{The ESP auction}
	\begin{itemize}
		\item Input: bid levels $\mathcal{B}$ and a bidding profile $(b_1,b_2,\ldots,b_n)$.
		\item Let $b^*$ denote the highest bid, $b'$ denote the second highest bid.
		\item the allocation rule and the payment rule are as follows.
		\begin{itemize}
			\item If $b^*=b'$, the item is given to the highest bidders with equal probability and the winner is charged $b^*$.
			\item If $b^*\neq b'$, the item is given to the highest bidder. His payment depends on the number of bidders who bid $b'$, denoted by $n(b')$. Let $b''$ be one level above $b'$. The payment of the winner is $\frac{n(b')b''+b'}{n(b')+1}$.
			\item Losers get nothing, pay nothing.
		\end{itemize}
	\end{itemize}
\end{algorithm}
\textbf{Remark:}

$(b',b'',b^*)$ can also be $(l_k,l_{k-1},l_i)$.
The $\frac{n(b')b''+b'}{n(b')+1}$ is in fact an indifference point.
If we charge more than this amount, the winner has incentive to underbid; otherwise, some bidder may have incentive to overbid.
\subsection{Incentive compatibility}

We first show that our auction is dominant strategy truthful, in the sense that each bidder finds in her best interest to report the highest bid level below (or equal to) her type.

\begin{theorem}
	The ESP auction is truthful.
\end{theorem}

\begin{proof}

	Our proof will discuss two cases.
	First, we show that agents cannot benefit from overbidding.
	Then, we show that they cannot benefit from underbidding either.
	
	We consider an arbitrary bidder $i$, whose type is $v_i$ and the highest bid level under $v_i$ is $b_i^*$.
	In the following discussion, we compare to the truthful case where $i$ bids $b_i^*$ and show that overbidding or underbidding cannot increase $i$'s utility.
	\begin{itemize}
		\item First, we argue that $i$ cannot benefit from reporting higher than $b_i^*$.
		\begin{itemize}
			\item If $i$ is the only possible winner\footnote{$i$ bids strictly higher than the second highest bidder.}, report higher will not reduce her payment.
			\item If someone else bid $b'$ higher than $v_i$, $i$ needs to bid at least $b'$ to be possible to get the item, and pay at least $b'$. As $b'>v_i$, $i$ will lose money.
			\item If $i$ is tied with $k(k\geq 1)$ bidders at the top bid, $i$ gets $(v_i-b_i^*)/(k+1)$ in expectation. If $i$ reports higher, $i$ will be the winner, but charged $(b_i' k+b_i^*)/(k+1)$, where $b_i'$ is the level closest above $b_i^*$. It results in a lower utility since $$v_i-(b_i' k+b_i^*)/(k+1)<(v_i-b_i^*)/(k+1).$$
		\end{itemize}
		\item Then we argue that $i$ cannot benefit from reporting lower than $b_i^*$.
		\begin{itemize}
			\item If $i$ is not the only possible winner, underbid disqualifies $i$ as a winner. Since obviously $i$ gets non-negative utility whenever he is a winner, underbid will not increase her utility in this case.
			\item Now, we consider the case where $i$ is the only possible winner. If $i$ reports less but higher than  the second highest bid. $i$'s utility remains the same. If $i$ reports less than the second highest bid, $i$ will no longer get the item. If $i$ bid the same as the second highest bid, $i$'s utility will become  $(v_i-b_{sec})/(k+1)$, which is less than the original utility $v_i-(b_{sec}' k+b_{sec})/(k+1)$. Here $b_{sec}$ is the second highest bid, $k$ is the number of bidders who bid $b_{sec}$ and $b_{sec}'$ is the bid level closest above $b_{sec}$
		\end{itemize}
	\end{itemize}
	
	To sum up, we have shown that overbid or underbid does not increase the utility of an agent.
\end{proof}
\begin{example}
	Figure \ref{fig:ds} plots the utility of a bidder in a specific instance. When her valuation is between 1 and 2, 1 is the optimal bid level to choose; when her valuation is above 2, bidding 2 is a better choice.
\end{example}
\begin{figure}
	\centering
	% Requires \usepackage{graphicx}
	\includegraphics[width=0.6\linewidth]{ds.pdf}\\
	\caption{In an auction with 3 bidders and two bid levels $\{1,2\}$. The first two bidders bid 1. This figure shows the third bidder's utility when choosing different bid levels.}\label{fig:ds}
\end{figure}

\subsection{Non-dominated in welfare and revenue}

%Another important property of ESP mechanism is pointwise non-dominance.
In this subsection, we first show that, in our setting, there is no mechanism that satisfies the standard efficiency, i.e., always allocate the item to the bidder with the highest valuation.
Then, we prove that ESP mechanism is not dominated in social welfare and revenue (among all truthful and IR mechanisms).
%Finally, we get a result on pointwise non-dominant on revenue among all DSIC, IR mechanisms.

\begin{theorem}\label{thm:ne}
	In our setting, there does not exist an efficient mechanism.
\end{theorem}

\begin{proof}
%	In this proof, we simply consider a setting with 2 bidders.
%	Consider two type profiles $\{v_1,v_2\}$ and $\{v_1',v_2'\}$, where $v_1'>v_2'>v_1>v_2$.
%	
%	First, we claim that under an efficient auction, the two type profiles should corresponds to two different bid profiles.
%	We prove this by contradiction.
%	If they correspond to the same bid profile, $v_1$ and $v_1'$ must correspond to the same bid level for bidder $1$, so do $v_2$ and $v_2'$ for bidder $2$.
%	So, the profile $\{v_1,v_2'\}$ must correspond to the same bid profile as $\{v_1,v_2\}$, but different winner, a contradiction.
%	
%	Therefore, there must be infinite number of such pairs $\{v_1^i,v_2^i\}$ such that for each $i>j$, $v_1^i>v_2^i>v_1^j>v_2^j$.
%	All these value profiles should be mapped to different bid profiles. Since the number of bid levels are finite, such mapping does not exist.
%	So There does not exist a mechanism which is always efficient.

	We assume by contradiction that there exists an efficient mechanism $\mathcal{M'}$ (that always allocates the item to the bidder with the highest type). 
	%We show that the mechanism cannot always tell which bidder has the highest value.
	
	We consider the case with only two bidders (can easily extend to $n$ bidder case by assuming others have zero valuation). Given the mechanism $\mathcal{M'}$ and a type profile $(v_1,v_2)$, by assumption, we must know that $\mathcal{M'}$ allocates the item to bidder $d$ such that (1) if $v_1<v_2$, $d=2$; (2) if $v_1>v_2$, $d=1$; (3) if $v_1=v_2$, $d$ can be either 1 or 2.
	
	We now consider two type profiles $(v_1,v_2)$ and $(v_1^1,v_2^1)$, where $v_1^1>v_2^1>v_1>v_2$. Case (1) If, in any Bayes Nash equilibrium $(s_1,s_2)$, the two value profiles correspond to the same bid profile (i.e., $s_1(v_1)=s_1(v_1^1)$ and $s_2(v_2)=s_2(v_2^1)$). We must have $\mathcal{M'}$ allocates to bidder $1$ (by efficiency of $\mathcal{M'}$). Moreover, we have the type profile $(v_1,v_2^1)$ corresponds to the same bid profile $(s_1(v_1),s_2(v_2^1))=(s_1(v_1),s_2(v_2))$. As a result, $\mathcal{M'}$ allocates to bidder 1, contradicting to the efficiency of $\mathcal{M'}$. Case (2) If $(v_1,v_2)$ and $(v_1^1,v_2^1)$ correspond to different bid profiles, we add them to a set $T$ that is initialized to be empty. Clearly, since the valuation domain is continuous, we can easily construct an infinite series of  $\{(v_1^t,v_2^t)|t=1,2,\ldots\}$, such that for any $t\geq 2$ and any $k<t$, $v_1^t>v_2^t>v_1^k>v_2^k$. The two profiles $(v_1^t,v_2^t)$ and $(v_1^k,v_2^k)$ must be belong to Case (2) for all $t$  and $k$ (otherwise, the same contradiction occurs as in Case (1)). In other words, we have created an infinite set of type profiles where any two profiles have different bid profiles. Since the total number of bid profiles is finite, we have a contradiction.

\end{proof}

As shown in Theorem \ref{thm:ne}, standard efficiency is not possible in our setting. We apply a weaker notion of efficiency and show that ESP meets this requirement.
\begin{theorem}\label{thm:exp}
	The ESP auction is not dominated on social welfare.
\end{theorem}
%\begin{figure}
%	\caption{Cases discussed in the the proof of Theorem \ref{thm:exp}}
%	\label{fig:pr3}
%	\centering
%	\includegraphics[width=0.7\linewidth]{proof3.pdf}
%\end{figure}
\begin{proof}
	We assume by contradiction that there exists a mechanism $\mathcal{M'}$ that is more efficient than the ESP mechanism $\mathcal{M^*}$ for any instance. First, we show that if $\mathcal{M'}$ dominates $\mathcal{M^*}$, $\mathcal{M'}$ needs to use as many bid levels as $\mathcal{M^*}$. Then, we consider the optimal strategy of bidders when their valuation is exactly a bid level. Finally, we found that $\mathcal{M^*}$ and $\mathcal{M'}$ have identical allocation rules, which contradicts to that $\mathcal{M'}$ dominates $\mathcal{M^*}$.
	
	Let $s_{i,\mathcal{M}}: [0,v_{max}]\rightarrow \mathcal{B}$ be the BNE strategy of agent $i$ under mechanism $\mathcal{M}$. For any $i$, if $v\in(b_l,b_{l+1})$, $s_{i,\mathcal{M^*}}(v)=b_l$; if $v=0$, $s_{i,\mathcal{M^*}}(v)=0$; if $v=l_k$ and $v\neq 0$, $s_{i,\mathcal{M^*}}(v)=l_k~or~l_{k+1}$.

	
	\begin{lemma}\label{lmm:dif}
		If $s_{i,\mathcal{M^*}}(v')\neq s_{i,\mathcal{M^*}}(v'')$, then $s_{j,\mathcal{M'}}(v')\neq s_{j,\mathcal{M'}}(v'')$ for any $j$ and $v',v''\notin \mathcal{B}$.
	\end{lemma}
	\begin{proof}
		Without loss of generality, we assume $v'>v''$.  For contradiction, we assume $s_{j,\mathcal{M'}}(v')= s_{j,\mathcal{M'}}(v'')$. 
		For any agent $i\neq j$, let $v_i=v''+\epsilon$, such that $s_{i,\mathcal{M^*}}(v''+\epsilon)= s_{i,\mathcal{M^*}}(v'')$.
		The existence of $\epsilon$ is ensured by the condition that $v',v''\notin \mathcal{B}$.
		Let $v_j=v'$, $\mathcal{M^*}$ knows that $j$ has the highest valuation from the bids but $\mathcal{M'}$ does not know.
		A contradiction.
	\end{proof}
	
	For any bidder $i$, $s_{i,\mathcal{M^*}}(v)$ can be any element in $\mathcal{B}$ when $v\in[0,v_{max}]\backslash \mathcal{B}$. By Lemma \ref{lmm:dif}, $s_{i,\mathcal{M'}}$ has at least $|\mathcal{B}|$ different possible values.
	All these values should be in $\mathcal{B}$. So, for any $v\in[0,v_{max}]\backslash \mathcal{B}$, there is a one-on-one mapping $g:\mathcal{B}\rightarrow\mathcal{B}$, such that $s_{i,\mathcal{M^*}}(v)=b'$ if and only if $s_{i,\mathcal{M'}}(v)=g(b')$.
	
	Now, we consider the case where the valuation of a bidder is exactly on a bid level $l_k~(k<n)$. 
	We assume when $l_{k-1}>v>l_k$ (Let $l_0$ be $v_{max}$), $s_{i,\mathcal{M'}}(v)=l_{u}$; when $l_{k}>v>l_{k+1}$, $s_{i,\mathcal{M'}}(v)=l_{t}$.
	We can easily extend the proof of Lemma \ref{lmm:dif} to prove that $s_{i,\mathcal{M'}}(l_k)$ must be in $\{l_u,l_t\}$.
	Then, we are going to show that $\mathcal{M'}$ cannot dominate $M^*$ no matter how agent $i$ (selecting $l_u$ or $l_t$).
	We consider an instance with only two positive-value bidders $i$ and $j$. We use \textit{up} to denote the behavior that an agent with value $l_k$ bids $l_u$ in $\mathcal{M'}$ or $l_k$ in $\mathcal{M^*}$, otherwise \textit{down} (bid $l_t$ in $\mathcal{M'}$ or $l_{k+1}$ in $\mathcal{M}$). The behaviors of $i$ and $j$ under a certain mechanism forms a behavior profile, such as (up,down), which means that $i$ chooses up and $j$ chooses down. Then, Table \ref{tab:pr3} lists the instances where $\mathcal{M'}$ is dominated by $\mathcal{M^*}$. We conclude that $\mathcal{M'}$ is not dominated by $\mathcal{M^*}$ on any instance only when the behavior profiles are the same.
	\begin{table}
		\tiny
		\caption{Type profiles where $\mathcal{M'}$ is dominated by $\mathcal{M^*}$. (Let $\epsilon<min(l_k-l_{k+1},l_{k-1}-l_k)$)}
		\label{tab:pr3}

				\begin{tabular}{c|c|c|c|c} \hline
					
					\backslashbox{$\mathcal{M^*}$}{$\mathcal{M'}$} & (up,up) & (up,down) & (down,up) & (down,down)\\ \hline
					(up,up)&-&$(l_k-\epsilon,l_k)$&$(l_k,l_k-\epsilon)$&$(l_k-\epsilon,l_k)$\\ \hline
					(up,down)&$(l_k+\epsilon,l_k)$&-&$(l_k,l_k-\epsilon)$&$(l_k,l_k-\epsilon)$\\ \hline
					(down,up)&$(l_k,l_k+\epsilon)$&$(l_k-\epsilon,l_k)$&-&$(l_k-\epsilon,l_k)$\\ \hline
					(down,down)&$(l_k,l_k+\epsilon)$&$(l_k,l_k+\epsilon)$&$(l_k+\epsilon,l_k)$&-\\ \hline
					
				\end{tabular}
			\end{table}
	
	Specially, when $v_i=l_n=0$, the proof of Lemma \ref{lmm:dif} can be used to verify that $s_{i,\mathcal{M'}}(0)$ should be equal to $s_{i,\mathcal{M'}}(v)$ where $0<v<l_{n-1}$.
	
	Above all, we have shown that $\mathcal{M'}$ has exactly the same information as $\mathcal{M^*}$. As $\mathcal{M^*}$ has made full use of its information to allocate the item efficiently. $\mathcal{M'}$ cannot be more efficient. A contradiction. 
	
	
	
	
	
%	it's easy to know that (1)for any two agents $i,j$ and any bid level $b'$, (2) if only when $v\in [v^1,v^2)$, $\mathcal{A}_{i,\mathcal{M}}(v)=b'$ and (3) $ b''\in \{\mathcal{A}_{j,\mathcal{M'}}(v)|v\in [v^1,v^2)\}$, we have that $b''$ cannot be the optimal bid for $j$ when $v_j\notin [v^1,v_2)$. Or in other words, each color in $i$'s bar (as in Figure \ref{fig:pr3}) corresponds to at least one color in $j$'s bar. 
%	
%	If the bars for $\mathcal{A}_{i,\mathcal{M^*}}$ and $\mathcal{A}_{j,\mathcal{M'}}$ are not the same, it should be like Figure \ref{fig:pr3}(c), otherwise $\mathcal{M}$ is the same as $\mathcal{M'}$. 
%	However, the bar for $\mathcal{A}_{i,\mathcal{M^*}}$ uses up all colors (bid levels), so the bar for $\mathcal{A}_{j,\mathcal{M'}}$ is infeasible, which makes a contradiction.
%	 
	
\end{proof}

The following theorem demonstrates that ESP also has a desired property on revenue.

\begin{theorem}
	The ESP auction is not dominated in revenue among all truthful, IR auctions.
\end{theorem}

\begin{proof}
	%We prove this theorem by contradiction.
	We assume that a mechanism $\mathcal{M'}$ gets no less revenue than the ESP $\mathcal{M}$ for all type profiles.
	Then,
	\begin{itemize}
		\item $\mathcal{M'}$ should always sell out the item unless all the bidders bid 0, because except for this case, $\mathcal{M}$ yields a positive revenue.
		When all the bidders bid 0, $\mathcal{M'}$ should charge 0. Otherwise, if the type profile is all 0, $\mathcal{M'}$ will not be IR.
		
		\item when the highest two bids are the same, $\mathcal{M'}$ must charge their bid. If $\mathcal{M'}$ charges even higher, one can construct an instance that all the types are exactly the top bid where  $\mathcal{M'}$ is not IR;
		if $\mathcal{M'}$ charges lower, it will yield less revenue.
		
%		\item given the second highest bid fixed, $\mathcal{M'}$ should charge the same amount of money whatever the highest bidder bids only if her bid is higher than the second highest bid. Otherwise, the winner may manipulate.
		\item for three bid levels $l_i>l_j>l_k$, the following two cases should charge the same amount of money: (1) $l_i$ and $l_k$ are the top two bids and (2) $l_j$ and $l_k$ are the top two bids.
		If case (1) charges more, the winner in case (1) can report $l_j$ to reduce her cost; otherwise, the winner in case (2) can report $l_i$ to reduce her cost.
		
		\item When the highest two bids are on neighbor levels. $\mathcal{M'}$ should charge the same as $\mathcal{M}$, as it is the only point to avoid manipulating.
		Suppose only one bidder bids the highest bid $l_i$ while $k$ bidders bid $l_{i+1}$.
		The revenue by $\mathcal{M}$ is denoted by $opt=(k l_i+l_{i+1})/(k+1)$.
		\begin{itemize}
			\item If $\mathcal{M'}$ charges the winner more than $opt$ and the type of the winner is $l_i$, the winner can bid $l_{i+1}$ to get a higher revenue.
			\item If $\mathcal{M'}$ charges the winner $opt-t (t>\frac{k}{k+1}\epsilon>0)$. We consider another case that $k+1$ bidders' truthful bid is $l_{i+1}$, while exactly one of them has a type more than $l_i-\epsilon$, then bidding $l_i$ can increase her utility.
		\end{itemize}
		
		
	\end{itemize}
	With all these properties, $\mathcal{M'}$ gets exactly the same revenue as $\mathcal{M}$.
\end{proof}

\subsection{Revenue comparison between ESP and SP}

A well studied auction mechanism is the second price auction (SP), where the auctioneer charges the highest bidder the second highest bid.
The key difference between ESP and SP is that when the highest bid differs from the second, ESP charges more than SP.
%If we directly apply SP in our setting, truthfully reporting may not be the best strategy for a bidder.
%However, as the bidders typically don't know the bids of the others and truthful reporting ensures the loss of each agent is bounded by the maximum difference between neighboring bid levels, it is   reasonable to assume bidders would like to bid truthfully.

We remark that bidding truthfully may not be a dominant strategy for the SP in our setting.
However, we still regard the truthful strategy as a sensible strategy in the SP for two reasons (1) the equilibrium of the SP in our setting might be hard to compute (2) For practical second price auction with discrete bid levels such as {\em eBay}, the default proxy bidding strategy is to bid truthfully.

For the reasons above, we compare the revenue of the ESP and SP, when bidders are truthful. We show, the ESP can yield extra revenue compared to the SP.
The extra revenue is bounded by the maximum gap between two consecutive bid levels.
When the distribution of the types are i.i.d., the difference can be up to $\frac{1}{e}$ times the difference between the highest and the lowest bid levels and this bound is tight.



\begin{theorem}
	Given the distributions of bidders' types are i.i.d. and all the bidders truthfully report, the revenue of ESP can be at most $\frac{1}{e}MG$ more than SP, where $MG$ is the difference between the highest and the lowest bid levels. 
\end{theorem}
\begin{proof}
	In this proof, the difference between the expected revenues of the two auctions is called \textit{revenue gap}.
	Our proof composes of two parts.
	First, we show that ESP can't get more extra revenue than $\frac{1}{e}MG$.
	Second, we show that the extra revenue can be arbitrary close to $\frac{1}{e}MG$.
\begin{itemize}
	\item \textbf{Upper bound: }The overall idea is that when every bidder truthfully reports, first price auction (FP) gets more or the same revenue compared to ESP.
	We show that the difference of revenues between the first price auction and the second price auction is bounded by $\frac{1}{e}MG$.
	It implies that the revenue gap between ESP and SP is bounded by $\frac{1}{e}MG$. 
	FP gets more revenue than SP only when the highest bid is not the same as the second highest bid.
	$Pr(b_i,b_j)$ denotes the probability that $b_i$ is the highest bid and $b_j$ is the second highest bid.
	The revenue difference between FP and SP is that
	\begin{eqnarray}
	\scriptsize
	&&\sum_{b_i,b_j\in \mathcal{B}}Pr(b_i,b_j)(b_i-b_j)\nonumber\\
	&=&\sum_{k\in \{1,2,\ldots,n-1\}}\sum_{b_i\geq l_k,  b_j<l_k}Pr(b_i,b_j)(l_k-l_{k+1})\nonumber\\
	&=&\sum_{k\in \{1,2,\ldots,n-1\}}Pr(v\geq l_k)(1-Pr(v\geq l_k))^{m-1}\nonumber\\
	&& m(l_{k}-l_{k+1})\nonumber\\
	&\leq&\sum_{k\in \{1,2,\ldots,n-1\}}\max_{0\leq p\leq 1}p(1-p)^{m-1}m(l_k-l_{k+1})\nonumber\\
	&=&\frac{1}{e}\sum_{k\in \{1,2,\ldots,n-1\}}(l_k-l_{k+1})\nonumber\\
	&=&\frac{1}{e}MG\nonumber
	\end{eqnarray}
	So, the revenue gap between FP and SP is bound by $\frac{1}{e}MG$.
	Thus, the revenue gap between ESP and SP is bounded by $\frac{1}{e}MG$.
	\item \textbf{Tightness:} Consider a special case, where there are $n$ bidders.
	Each bidder has a type $2$ with $p$ probability and $1$ with $1-p$ probability.
	Bid levels consist of two levels $\{1,2\}$.
	When all the bidders truthfully bid, each bidder bids $2$ with $p$ probability and bids $1$ with $1-p$ probability.
	ESP and SP yield different revenues only when only one bidder bids $2$ and the others bid $1$.
	The probability of this case is $mp(1-p)^{m-1}$.
	Under such condition, ESP gets $\frac{m-1}{m}$ more revenue.
	So, in expectation, the revenue gap is $mp(1-p)^{m-1}\frac{m-1}{m}$.
	Set $p$ to be $\frac{1}{m}$, when $m$ tends to infinity, the revenue gap becomes $\frac{1}{e}$.
\end{itemize}	
\end{proof}

\section{Bid level selection}\label{sec:sel}

A natural problem to study is, under ESP, given the distributions of valuations, how to select a subset of bid levels to optimize revenue?
We consider two cases, selection from a given set and selection from the valuation domain. In order to select the optimal subset of bid levels, in this section, we assume that the seller knows the distribution of the bidders' valuations.
Because of the page limitation, details of this section are omitted and can be found in the full version.


%It's quite hard to get a closed form for both cases, which is also confirmed by our experimental results.
\begin{figure*}[t]
	\centering
	\begin{minipage}[t]{0.5\linewidth}
		\centering
		\includegraphics[width=3in]{n3.pdf}
	\end{minipage}%
	\begin{minipage}[t]{0.5\linewidth}
		\centering
		\includegraphics[width=3in]{n5.pdf}
	\end{minipage} 
	\begin{minipage}[t]{0.5\linewidth}
		\centering
		\includegraphics[width=3in]{n7.pdf}
	\end{minipage}%
	\begin{minipage}[t]{0.5\linewidth}
		\centering
		\includegraphics[width=3in]{n9.pdf}
	\end{minipage} 
	\caption{These figures plot the optimal bid levels, expected second highest value (SV) and corresponding revenues. We list the cases where $n=3,5,7,9$.}
	\label{fig:level}
\end{figure*}

\subsection{The dynamic programming approach}

Proper selection of bid levels is extremely important under some special distributions.
For example, let $\mathcal{B}=\{0,0.5,1\}$, and the values of bidders are distributed only on 0 and 1.
Obviously, removal of 0.5 can increase the revenue.

To address this issue, we introduce an algorithm via dynamic programming.
The key observation is that the bids below the second highest bid are irrelevant in our mechanism.
The general idea is as follows.
We scan all the bid levels from high to low.
At each bid level $l_i$, we get the maximum revenue for all the cases where the second highest bid is no less than $l_i$.
At the same time, we record the subset of bid levels corresponding to the maximum revenue.
When we finished the algorithm, we can get the optimal bid levels and the corresponding revenue.
%\begin{algorithm}[ht]
%	\label{alg:dp}
%	\caption{The algorithm for selecting bid levels}
%	Input:
%	\begin{itemize}
%		\item The bid levels $\mathcal{B}$.
%		\item The distribution of all the bidders $\mathcal{V}\in R^{m n}$, $V_{ij}$ denotes the probability that bidder $j$'s type is between $l_{i-1}$ and $l_i$ (Specially, $l_0=+\infty$).
%	\end{itemize}
%	Output: $\mathcal{B'}\subset\mathcal{B}$, where $\mathcal{B'}$ is the optimal bid levels.\\
%	\begin{itemize}
%		%\item Sort all the bid levels from high to low and add an 0 at the end, denoted by $s_1,s_2,\ldots,s_n,s_{n+1}=0$.
%		\item $pre[0] \leftarrow -1$. Element $pre[i]$ in the array $pre$ denotes the closest left bid level before $l_i$ when given the bid level $l_i$ is selected and the revenue (for all the instances when the second highest value is higher than $l_i$) is maximized.
%		\item $rev[0]\leftarrow0$. $rev[i]$ denotes the maximum revenue among all the cases where the second highest value is at least $l_i$
%		\item For $i\leftarrow 1,2,\ldots,n+1$,
%		\begin{itemize}
%			\item $rev[i]\leftarrow\max_{0\leq j<i}(rev[j]+incre(j,i))$, where $incre(j,i)$ denotes the expected revenue when all the bid levels between $j$ and $i$ are removed and the second highest value is less than $l_j$ and no less than $l_i$. $incre(j,i)$ can be obtained from the distribution of values $\mathcal{V}$.
%			\item $pre[i]\leftarrow\arg_j\max(rev[j]+incre(j,i))$
%		\end{itemize}
%		\item $curr\leftarrow pre[n+1]$, $levels=\{\}$
%		\item While $curr\neq 0$,
%		\begin{itemize}
%			\item add $l_curr$ to $levels$.
%			\item $curr\leftarrow pre[curr]$.
%		\end{itemize}
%		\item Return $levels$.
%	\end{itemize}
%\end{algorithm}

\subsection{A non-linear programming approach}
\label{sec:opt}

The algorithm above solved the level selection from a given set. A further question is how to select a subset of bid levels from a continuous domain.
In this case, the algorithm above can be utilized to approximate the optimal levels by selecting a dense subset of bid levels.

In this subsection, we propose another approach to get the optimal bid levels, which is accurate but not scalable.
The general idea is that we solve a non-linear program with the revenue as the objective function subject to the order and domain of the bid levels.

Let $\{p_i|i=0,1,2,\ldots,n\}$ denote variables for bidding levels $\{v_{max}, l_1,l_2,\ldots,l_n\}$, 

$Rev_1(i)$ denotes the expected revenue when the top two bids are both $p_i$
\begin{eqnarray}
&Rev_1(i)=p_i(\prod_{1\leq j\leq m}Pr(b_j<p_{i-1})-\prod_{1\leq j\leq m}Pr(b_j<p_i)&\nonumber\\
&-\sum_{1\leq j\leq m}Pr(b_j\geq p_i)\prod_{1\leq k\leq m,k\neq j}Pr(b_k< p_i))&\nonumber
\end{eqnarray}
$Rev_2(i,k)$ denotes the expected revenue when the highest two bids are different and $k$ bidders bid the second highest bid $p_{i+1}$.
\begin{eqnarray}
\small
Rev_2(i,k)&=&\frac{p_i k+p_{i+1}}{k+1}\sum_{1\leq j\leq m}Pr(b_j>p_i)\nonumber\\
&&\sum_{a_1,a_2,\ldots, a_k\in \{1,2,\ldots,n\}\backslash\{j\}}\nonumber\\
&&(\prod_{u\in 1,2,\ldots, k}Pr(p_i\geq b_{a_u}\geq p_{i+1}))\nonumber\\
&&(\prod_{u\in \{1,2,\ldots,n\}\backslash\{j,a_1,a_2,\ldots,a_k\}}Pr(b_u<p_{i+1}))\nonumber
\end{eqnarray}
Thus, the total expected revenue is as follows.
\begin{eqnarray}
& & obj(p_0,p_1,p_2,\ldots,p_n)\nonumber\\
&=&\sum_{i=1,2,\ldots,n}Rev_1(i)+\sum_{i=1,2,\ldots,n}\sum_{j=1,2,\ldots,m-1}Rev_2(i,j)\nonumber
\end{eqnarray}
The optimization program is shown below.
\begin{equation*}
\begin{aligned}
& \text{Maximize}
& & obj(p_0,p_1,p_2,\ldots,p_n) \\
& \text{Subject to}
& & p_i\geq p_{i+1},\forall i\in\{0,1,\ldots,n-1\}\\
&&& p_0=v_{max}\\
&&& p_{n}=0
\end{aligned}
\end{equation*}
%in Algorithm \ref{alg:sim}.
%\begin{algorithm}\label{alg:sim}
%	\caption{The non-linear program for solving the optimal bid levels}
%	\begin{itemize}
%		\item Maximize $obj(p_0,p_1,p_2,\ldots,p_n,p_{n+1})$
%		\item Subject to:
%		\begin{itemize}
%			\item For all $i=1,2,\ldots,n$, $p_i\geq p_{i+1}$
%			\item $p_0=+\infty$ and $p_{n+1}=0$
%		\end{itemize}
%	\end{itemize}
%\end{algorithm}

\section{Experiments}

Previously, we introduced a non-linear program for finding optimal bid levels.
We apply it to uniform distribution as a case study.
Our results demonstrate the feasibility of optimizing bid levels in simple cases.
Actually, other work for optimal bidding level selection also applies non-linear program\cite{li2011revenue}.

We assume that the types of all the bidders are independent and uniformly randomly distributed on interval [0,1].
For any bidder $i$ with bid $b_i$ and any bid level $l_j$, $Pr(b_i\geq l_j)=1-l_j$, while $Pr(b_i<l_j)=l_j$.
Put the probabilities into the optimization program, we can get the optimal bid levels.


From the experimental results in Figure \ref{fig:level}, we have the following observations.
(1) The revenue is always between $l_1$ and $l_2$. 
(2) The curves of lower bid levels are steep. The curves of higher bid levels are smooth and monotonously increasing.
(3) The revenue curves of the last two figures are virtually the same.
(4) As the increasing of bid levels' amount, the revenue becomes closer to the expected second highest value ($\frac{m-1}{m+1}$). 


%In other words, we only need to set several bid levels to make the revenue almost maximized.

\section{Conclusion}
In this paper, we propose a truthful, individually rational and non-dominant mechanism for single-item auction.
Our mechanism makes bidders easier to choose their action without considerable loss of efficiency and revenue.
We propose algorithms for selecting optimal bid levels for ESP.
Our experimental results show that the revenue loss caused by discrete-bid is very small.
Also, our experiments show some interesting properties of the optimal bid levels under the uniform distribution.


\newpage
\bibliographystyle{named}
\bibliography{auction}

%\section*{Appendix}
%
%\appendix
%\section{A non-linear program for solving the optimal bid levels}
%In this subsection, we propose another approach to get the optimal bid levels, which is accurate but not scalable.
%First, we need to formally define our problem.
%
%We are given the probability density function of all the $m$ bidders, $f(i,v_i)$, denoting the probability density of bidder $i$ at type $v_i$.
%%Then, we are given an $n$, denoting the number of  bid levels.
%The goal is to return a optimal set of bid levels $\mathcal{B}$ that contains at most $n$ elements.
%
%The general idea is that we solve a non-linear program with the revenue as the objective function subject to the order and domain of the bid levels.
%In our optimization problem, the positions of bid levels are denoted by $p_1,p_2,\ldots,p_n$, satisfying $p_i>p_{i+1}$. \footnote{As $p_i$ denotes a variable in the optimization program, we don't use $l_i$.}
%In addition, we set $p_0=+\infty$ and $p_{n+1}=0$.
%First, we consider the case when the highest bid and the second highest bid are the same, say $p_i$.
%The winner needs to pay exactly their bid.
%The expected revenue of this case is:
%\begin{eqnarray}
%&Rev_1(i)=p_i(\prod_{1\leq j\leq m}Pr(b_j<p_{i-1})-\prod_{1\leq j\leq m}Pr(b_j<p_i)&\nonumber\\
%&-\sum_{1\leq j\leq m}Pr(b_j\geq p_i)\prod_{1\leq k\leq m,k\neq j}Pr(b_j< p_i))&
%\end{eqnarray}
%All the $Pr(.)$'s can be computed from the probability density function $f$.
%Then, we consider the case where the highest bid is at least $p_{i}$, while $k$ bidders bid the second highest bid $p_{i+1}$.
%In this case, the winner needs to pay $\frac{kp_i+p_{i+1}}{k+1}$ as shown in the mechanism.
%The expected revenue of this case is :
%\begin{eqnarray}
%\small
%Rev_2(i,k)&=&\frac{p_i k+p_{i+1}}{k+1}\sum_{1\leq j\leq m}Pr(b_j>p_i)\nonumber\\
%&&\sum_{a_1,a_2,\ldots, a_k\in \{1,2,\ldots,n\}\backslash\{j\}}\nonumber\\
%&&(\prod_{u\in 1,2,\ldots, k}Pr(p_i\geq b_{a_u}\geq p_{i+1}))\\
%&&(\prod_{u\in \{1,2,\ldots,n\}\backslash\{j,a_1,a_2,\ldots,a_k\}}Pr(b_u<p_{i+1}))\nonumber
%\end{eqnarray}
%
%
%Now, we are ready to show the optimization problem to solve the optimal bid levels.
%The expected revenue is the sum of the two cases above over all inputs.
%Only basic constraints are needed.
%
%By the derivation above, we are ready to give the optimization problem as follows.
%\begin{algorithm}
%	\caption{The non-linear program for solving the optimal bid levels}
%	\begin{itemize}
%		\item Maximize \\$\sum_{i=1,2,\ldots,n}Rev_1(i)+\sum_{i=1,2,\ldots,n}\sum_{j=1,2,\ldots,m-1}Rev_2(i,j)$
%		\item Subject to:
%		\begin{itemize}
%			\item For all $i=1,2,\ldots,n$, $p_i\geq p_{i+1}$
%			\item $p_0=+\infty$ and $p_{n+1}=0$
%		\end{itemize}
%	\end{itemize}
%\end{algorithm}
%
%When the density function is simple(for example uniform distribution), the optimization problem is possible to be solved.
%However, as the the rank of the objective function is $O(m)$, only small instances can be efficiently solved.
%Tools such as OpenOpt\footnote{\url{openopt.org}} can be used to solve such problems.

\end{document}


