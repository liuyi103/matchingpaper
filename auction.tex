%File: formatting-instruction.tex
\documentclass[letterpaper]{article}
\usepackage{ijcai15}
\usepackage{times}
\usepackage{url}
\usepackage{helvet}
\usepackage{courier}
\usepackage{graphicx}
\usepackage{subfigure}
\usepackage{slashbox}
\usepackage[lined,boxed,commentsnumbered, ruled]{algorithm2e}
\usepackage{balance}
\usepackage{amsmath}
\usepackage{xcolor}
\newtheorem{theorem}{Theorem}%[section]
\newtheorem{proposition}{Proposition}
\newenvironment{proof}{{Proof:}}{\hfill\rule{2mm}{2mm}}
\newenvironment{mechanism}{Mechanism}{\hfill\rule{2mm}{2mm}}
\newtheorem{definition}{Definition}
\newtheorem{example}{Example}
\newtheorem{lemma}{Lemma}
\newtheorem{property}{Property}

\title {Stochastic matching in simple settings}
\author {Someone}
 \begin{document}
% The file aaai.sty is the style file for AAAI Press
% proceedings, working notes, and technical reports.
%

\maketitle
\begin{abstract}
Motivated by applications like kidney exchange, recommendation system  and online dating, we study a matching problem with query-commit process.
In this setting, a centralized system allocates agents into pairs. If a pair of agents accept their assignments, they are matched and leave the market.
Otherwise, they will stay in the market.
Most previous work focus on giving worst case guarantee on different settings. 
However, surve
 


\end{abstract}

\section{Introduction}
\section{The setting}

Our problem can be modeled in an undirected graph $G=(V,E)$, $V=\{v_1,v_2,\ldots,v_n\}$ denotes the set of agents. $e_{ij}\in E$ denotes an edge between $v_i$ and $v_j$.
The goal of the matching is to maximize cardinality of matched agents.
In this paper, we consider two types of matching processes, 



\newpage
\bibliographystyle{named}
\bibliography{auction}

\end{document}


