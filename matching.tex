%File: formatting-instruction.tex
\documentclass[letterpaper]{article}
\usepackage{ijcai15}
\usepackage{times}
\usepackage{url}
\usepackage{helvet}
\usepackage{courier}
\usepackage{graphicx}
\usepackage{subfigure}
\usepackage{algorithm}
\usepackage{algpseudocode}
\usepackage{balance}
\usepackage{amsmath}
\usepackage{xcolor}
\newtheorem{theorem}{Theorem}%[section]
\newtheorem{proposition}{Proposition}
\newenvironment{proof}{{Proof:}}{\hfill\rule{2mm}{2mm}}
\newenvironment{mechanism}{Mechanism}{\hfill\rule{2mm}{2mm}}
\newtheorem{definition}{Definition}
\newtheorem{example}{Example}
\newtheorem{lemma}{Lemma}
\newtheorem{property}{Property}

\title {OptOrder: optimum serial dictatorship in stochastic matching}
\author {Someone}
 \begin{document}
% The file aaai.sty is the style file for AAAI Press
% proceedings, working notes, and technical reports.
%

\maketitle
\begin{abstract}
This paper studies a simplified version of the stochastic matching proposed in \cite{chen2009approximating}.
The stochastic matching problem is modeled in a graph $G=(V,E)$. 
Each edge $e_i$ has a probability to be existing. 
The market probs edges one by one (edge-probing setting).
If an edge exists, it will be matched and both ends will leave the market.
Otherwise, the edge can not be matched and one end of the edge will stay in the market only if she still has patience.
The goal is to match more existing edges.
Previous literature focus on finding bounds on settings with many parameters.
However, these bounds are usually meaningless in simple settings.
Even in simple settings, finding the optimum policy to match meets great difficulties mainly from two aspects.
First, there is no evidence that the optimal policy can be expressed in polynomial space.
Second, collecting the probabilities usually incurs considerable costs.

We study a novel setting where the market does not know whether an edge exists after matching and the patience is defined in a special way.
The goal in our setting is to find the optimum sequence for serial dictatorship that maximizes the number of matched edges.
We solve this problem via integer linear programming (ILP).
Our well designed ILP uses only $O(|V||E|)$ entries and outperforms the enumeration algorithm a lot.
Actually, our setting is related to a quite general setting  at a certain degree.
Although our work is for a quite specific setting, we move forward the first solid step towards ``optimum'' in the field of stochastic matching.

\end{abstract}

\section{Introduction}

An active line of study on matching problems has been conducted since 1970s~\cite{keylist} with applications on stable marriage problems, stable roommate problems, kidney exchange etc.
In traditional matching settings, properties like stability, strategy-proof and social welfare maximizing are desired.
However, during the last decade, inspired by applications such as kidney exchange and online dating, more attention has been attracted to stochastic settings.

\cite{chen2009approximating} defines a model ``stochastic matching'' and starts an active line of study on worst case bounds on related problems\cite{adamczyk2011improved,costello2012stochastic,poloczek2012randomized,goel2012matching,bansal2012lp}.
They share some common assumptions in their models.
The stochastic matching is modeled in a graph $G=(V,E)$.
Each edge $e$ has a probability to be existing.
Otherwise, the edge is fake.
There exists a matching system that can prob an edge to match.
When an existing edge is probed, it will confirm the match and will never be separated.
When a fake edge is probed, it will deny the match and the edge is removed from the graph.
The goal of the matching system is to maximize the number of matched edges or the total weight. 
Many variants have been studied in this setting, such as adding weights on vertices, probing an edge or a matching, setting the patiences, matching with eyes closed and matching in bipartite or general graphs\footnote{Often, more than one variants are usually considered in one paper at the same time, so we list the related work at the beginning of this paragraph.}.
All these study focus on finding bounds for the worst case compared to the optimum policy or the offline optimal solution.
However, for the most simple setting where all the existing probabilities are the same, no weights and no patiences setting, none of the algorithms above can be significantly better than serial dictatorship following a random order.
The current settings face two main drawbacks that prevent it from practical use.

First, to know the exact probability of a prob being accepted is difficult.
Typically, in order to know the exact  existing probabilities, the market needs to collect many data and conduct careful analyses on historical data.
It is worthy in some important markets like kidney exchange market~\cite{dickerson2013failure,dickerson2015futurematch}.
However, in many less important markets, collecting existing probabilities is quite costly and unworthy.
Similar to online dating problem, let us consider a friend recommendation problem.
Someone (called Alice) has many friends, $x$ of them are single boys and $y$ are single girls.
Alice wants to introduce boys to girls for marriage.
She wants to maximize the number of matched couples.
Every boy or girl has some requirements that must be satisfied.
However, even though a boy and a girl satisfy each other's requirements, they may still not be willing to match.
In this case, it is hard and costly for Alice to survey the acceptance ratio of each pair.
What Alice does have may be the estimated average acceptance ratios of each pair.
Such applications motivate us to study the settings where all the existing probabilities are unique.
Similar problems also appear in other applications like roommate problem~\cite{roth1982incentive}, but never serious considered to the best of our knowledge.

Second, as mentioned above, most study focus on giving worst case bounds. 
The main difficulty preventing people to desire the optimum solution is that the optimal solution maybe not able to be expressed in polynomial space.
Even though you have already gotten a sequence for serial dictatorship (note that the optimum solution may be be implementable by serial dictatorship), it is still hard to know the expected number of matched edges. 

Motivated by the two drawbacks above, we study a novel setting with identical existing properties for edges and a P-time solvable expression for the number of expected matched agents.
For short, we use the \textit{payoff} of a matching to denote the number of expected matched agents.
Our setting is inspired by boy/girl friends recommendation.
In our setting, the matching market probs one edge per time.
The existing probabilities of edges are the same.
We assume that the market cannot get feedbacks from the matched agents (vertices) in a short time because they are not willing to make it known to others.
We assume that each agents has two status \textit{active} or \textit{upset}.
Typically, an agent becomes upset because she is assigned to some matched agent and disappointed to the matching system.
When an agent becomes upset, she will always be upset and rejects all the probs. 
When an edge $e_{ij}$ (connecting two agents $v_i$ and $v_j$) is probed,
\begin{itemize}
	\item If the two agents $v_i$ and $v_j$ are unmatched, $e_{ij}$ is matched with probability $p$
	\item  If both $v_i$ and $v_j$ have been matched, nothing happens.
	\item If $v_i$ (or $v_j$) is unmatched but the other has been matched, $v_i$ becomes upset with probability $p$.
\end{itemize}
Our goal is to find the optimum probing sequence.

Actually, our setting is closely related to the setting of stochastic matching with equal existing probabilities, no patiences and no weights, called \textit{the general setting}. On the same graph, the payoff of our algorithm is also the lower bound of the optimum payoff of serial dictatorship in the general setting. 

The following part of this paper is organized as follows. Section 2 defines the serial dictatorship and show some fundamental good properties of it. Section 3 shows the construction of the integer linear programming step by step and prove its correctness. Section 4 explores methods to expand our algorithm to larger graphs. Section 5 shows our experimental results. Section 6 concludes this paper.


\section{The settings}

In this section, we will describe our model and the serial dictatorship. After that, some desired properties of serial dictatorship are discussed.

\subsection{The model}

Our problem is modeled in an undirected graph $G=(V,E)$, $V=\{v_1,v_2,\ldots,v_n\}$ denotes the set of agents. $e_{ij}\in E$ denotes an edge between $v_i$ and $v_j$. In the following analyses, elements in $E$ are renumbered to be $\{e_1,e_2,\ldots,e_{|E|}\}$. 
For any edge $e_{ij}\in E$, it has two states, \textit{existing} or \textit{fake}.
The probability of existing is a constant $p$.
The goal of the matching is to maximize cardinality. Each time, the system prob a pair of agents $v_i$ and $v_j$ with an edge $e_{ij}$ connecting them. 
\begin{itemize}
	\item If $e_{ij}$ exists, then $v_i$ and $v_j$ agree to match with each other, both of the two agents and the edges attached to them are removed from the graph. 
	The number of matched agents increases by 2.
	\item If $e_{ij}$ is fake, $v_i$ and $v_j$ refuses to match. The system removes $e_{ij}$ from $E$.
\end{itemize}


\subsection{ESD algorithm}

To maximize the number of matched agents, we apply \textit{edge serial dictatorship algorithm} (ESD).
The outline is described In Algorithm \ref{alg:esd}.

\begin{algorithm}
	\caption{ESD algorithm}
	\label{alg:esd}
	\begin{algorithmic}[1]
		\Require
		A sorted array of $E$, denoted by $S=s_1,s_2,\ldots,s_m$ and the set of vertices $V$.
		\Ensure A set of disjoint edges $A$, such that each two agents connected by an edge in $A$ agree to match. 
		\State $A=\emptyset$
		\For $i=1,2,\ldots,m$
		\State Let $v_x$ and $v_y$ be the two agents connected by $s_i$
		\If $v_x\in V$ and $v_y\in V$
		\State Prob $v_x$ and $v_y$ whether they agree to match with each other.
		\If Both $v_x$ and $v_y$ accept to match
		\State Put $s_i$ into $A$
		\State Remove $v_x$ and $v_y$ from $V$
		\EndIf
		\EndIf
		\EndFor
		\State Outputs $A$ as the set of matched edges.
	\end{algorithmic}
\end{algorithm}

Algorithm \ref{alg:esd} is a typical serial dictator algorithm. 
The system just picks edges one by one in some order (denoted by $S$) and try to match the current edge if both ends of the edge are still in the graph. 
Our goal is to maximize the expected size of the output set $A$ among all $S$.

\subsection{Desired properties}

We restrict our focus on the serial dictatorship for many reasons.

\textbf{Usability}. The solution of ESP has a size of $O(|E|)$, which equals to the size of input. The executor of the matching only needs to use O(1) computation for each prob. These two properties make the solution perfect to be used as handouts. In other words, the order can be generated by a computer and any people can execute the matching without a computer.

\textbf{Deterministic}. A great advantage of ESD is that it outputs a deterministic solution. One can count the expected number of matched agents easily. 
Many previous algorithms lack this property.
In our setting, given the serial $S$, the expected number of matched agents $c(S)$ can be counted by the following equation.
\begin{equation}\label{obj}
c(S)=2*\sum_{i=1}^m p(1-p)^{\sum_{j=1}^{i-1}\delta(s_j,s_i)}
\end{equation}

In equation \ref{obj}, $\delta(s_i,s_j)$ is an indicator of whether $s_j$ appears before $s_i$ and they have a common vertex. In other words, $\delta(s_j,s_i)=1$ only when $s_j$ is probed before $s_i$ and they have a common vertex, otherwise  $\delta(s_j,s_i)=0$. 
In the integer linear programming, we will use variables $\delta_{ij}$, which holds the same meaning as $\delta(s_i,s_j)$ here.

\section{Optimizing the serial}

Algorithm \ref{alg:esd} has given a framework of serial dictatorship. 
The problem is how to find the serial  $S$ that maximizing the expected cardinality of the matching.
In this section, we will introduce the integer linear program (ILP) we used to find the optimum $S$ step by step.

\subsection{Overview of the algorithm}

The objective function of our ILP is a representation of Equation \ref{obj}.
As for the constraints, if we faithfully write the constrains according to definition of serial dictatorship and the objective function, we need $O(|E|^4)$ variables in the ILP, which is insufferable. We are not going to discuss in details for this idea.

To reduce the number of variables, we introduce a novel method.
We first relax the solution space to obtain a relaxed-form ILP.
Then, we show that the solution of the relaxed-form ILP can be mapped to the optimal serial $S$.

The relaxed-form ILP is based on the following observation.
Given the optimal serial $S$, there is a determined order between any two edge.
For example, for $e_{ij}, e_{ik}\in E$, given $S$, we know that $e_{ij}$ is always probed before $e_{ik}$ or $e_{ik}$ is always before $e_{ij}$.
Now, we are ready to relax the solution space.
Rather than a total order, we only assume that there is a determined order between any two edges with a common vertex.
We do not have any other constraints on the solution space.
Then, we will introduce a method that maps the optimal solution of the relaxed-form ILP to the optimal serial $S$ in ESD algorithm.
The relaxed-form ILP only has $O(|E||V|)$ variables and $O(|E||V|)$ entries in total.
So, it improves the performance significantly.

\subsection{Variable Definition}

Three types of variables are used in our ILP\footnote{From here on, all the ``ILP'' refers to the relaxed-form ILP.}.
For a better presentation of our ILP, we give an alias for each element in $E$.
We number all the edges in $E$ as $e^1,e^2,\ldots,e^m$.
This notation is not conflict with the previous definition.
An edge $e^i$ may denote the same meaning as $e_{jk}$.
Now, we are ready to define the variables in our setting.
\begin{itemize}
	\item $\delta_{ij}(i,j\in\{1,2,\ldots,m\})$ denotes an indicator of $e_i$ is probed before $e_j$ and they have a common vertex (just as mentioned before). If both conditions are satisfied $\delta_{ij}=1$, otherwise $\delta_{ij}=0$. The value of $\delta_{ij}$ is selected from $\{0,1\}$.
	\item $o_i(i\in\{1,2,\ldots,m\})$ denotes the number of edges that are probed before $e_i$ and have a common vertex as $e_i$. By definition, we have $o_i=\sum_{j=1}^{m}\delta_{ji}$. This equation is also used as a constraint in the ILP. The value of $o_i$ is selected from nonnegative integers
	\item $l_{ij}(i,j\in\{1,2,\ldots,m\})$ denotes an indicator for whether $o_i$ is greater than or equal to $j$. This variable is used to express the objective function of ILP. Its value is selected from $\{0,1\}$.
\end{itemize}

\subsection{Objective function}

The objective of the ILP is to maximize to total number of cardinality, which has been shown in Equation \ref{obj}.
To express it in the ILP fashion, we write it as follows. The index ``2'' is omitted, thus the objective function becomes the expected number of matched edges.

\begin{eqnarray}
	obj&=&\sum_{i=1}^{m}p*(1-p)^{o_i}\label{simple}\\
	&= & p*\sum_{i=1}^{m} ((1-p)^{o_i}-(1-p)^{o_{i}-1}+(1-p)^{o_{i}-1}-\nonumber\\
	&&(1-p)^{o_{i}-2}+\ldots-(1-p)+(1-p)-1+1)\\
	&=& mp+p*\sum_{i=1}^{m}\sum_{j=1}^{o_i}((1-p)^j-(1-p)^{j-1})\\
	&=& mp+p*\sum_{i=1}^{m}\sum_{j=1}^{m}((1-p)^j-(1-p)^{j-1})l_{ij}\label{fobj}
\end{eqnarray}

In equation \ref{fobj}, the objective function becomes a linear function, such that it can be used in the ILP.
The current problem is how to make $l_{ij}$ an indicator for $o_i\geq j$.

As stated before, the solution of the ILP only make sense when the objective function $obj$ has been maximized.
So, we only need to add the following constraint to restrict $l_{ij}$.
\begin{equation}
o_i\leq j-1+l_{ij}*|E|\label{lcon}
\end{equation}

In our ILP, $l_{ij}$ only appears in the constraint above and the objective function.

\begin{lemma}\label{lmm}
	When $obj$ has been maximized, $o_i\geq j$ if and only if $l_{ij}=1$.
\end{lemma}
\begin{proof}
	To prove this lemma, we only need to show that the following two cases are impossible when $obj$ is maximized.
	\begin{itemize}
		\item $o_i<j$ and $l_{ij}=1$.
		\item $o_i>j$ and $l_{ij}=0$.
	\end{itemize}
	The second case is obviously impossible by Equation \ref{lcon}.
	
	If the first case is satisfied and $obj$ is maximized, set $l_{ij}$ to be 0 will increase the $obj$ and all the constraints are still satisfied. A contradiction.
\end{proof}

\subsection{The ILP}
Now we are ready to conclude the ILP we used to solve the optimal serial as follows.

\begin{eqnarray}
\text{Maximize:}&& mp+p*\sum_{i=1}^{m}\sum_{j=1}^{n_i}((1-p)^j-(1-p)^{j-1})l_{ij}\label{stobj}\\
\text{Subject to:}&& 1.~\delta_{ij}+\delta_{ji}=1, \forall e_i,e_j\in E, e_i\cap e_j\neq \emptyset \label{con1}\\
&& 2.~o_i\leq j-1+l_{ij}*|E|,  i=1,2,\ldots,m, j=1,2,,\ldots,n_i\label{ni}\\
&& 3.~o_i=\sum_{e_j\cap e_i\neq \emptyset,i\neq j}\delta_{ji}, \forall i=1,2,\ldots,m \label{oo}
\end{eqnarray}

 $e_i\cap e_j\neq \emptyset$ denotes that $e_i$ and $e_j$ have a vertex in common. 
In equation \ref{ni} and the objective function, $n_i$ denotes the number of edges that have a common vertex as $v_i$.
Introducing $n_i$ decreases the number of variables we used.
Further, we omitted the $\delta_{ij}$'s for disjoint $e_i$ and $e_j$.
These two steps ensure the number of entries to be $O(|E||V|)$.
We will discuss the complexity problem later.

\subsection{Mapping to ESD}
We note that when the objective function is not maximized, the values of variables correspond to nothing useful for serial dictatorship.
However, when the objective function, things become different.
Now, we are going to show how to map the solution of the ILP to the optimal sequence $S$ for ESD.
The method is as follows.

\textbf{Mapping process: }First, we define a function $o$ satisfying $o(e_i)=o_i^*$. 
Note that we reuse the $o$ for the same meaning.
$o_i^*$ denotes the value of $o_i$ when the objective function in the ILP has been maximized.
Then, we sort the $E$ to be a sequence $S^*=s_1^*,s_2^*,\ldots,s_m^*$, satisfying $\forall i,j\in\{1,2,\ldots,m\}$, $o(s_i)<o(s_j)$.
The $S$ is the optimal sequence for ESP.

\subsection{A summary of our algorithm for optimum serial dictatorship}
For a better understanding, our algorithm for optimum serial dictatorship is shown in Algorithm \ref{alg:full}.
The overall idea is to solve the ILP, map to a sequence and run ESD algorithm.
\begin{algorithm}
	\caption{Optimum serial dictatorship}
	\label{alg:full}
	\begin{algorithmic}[1]
		\Require
		A graph $G=(V,E)$
		\Ensure A set of disjoint edges $A$, such that each two agents connected by an edge in $A$ agree to match and the expected size of $A$ is maximized among solutions from edge serial dictatorship. 
		\State Run the ILP shown in Equations \ref{stobj}-\ref{oo} on graph $G$. The value profile for the optimum solution is denoted by $P$. 
		\State Map $P$ to a sequence $S$ by mapping process. 
		\State Run Algorithm \ref{alg:esd} on $S$, outputs the matched edges $A$ as output.
	\end{algorithmic}
\end{algorithm}

\subsection{The correctness proof}
Recall the definition of $o_i$, we just sort all the edges according to the number of neighboring edges probed before the edge.
For short, we call ``the number of neighboring edges probed before the edge'' as the \textit{order} of the edge.
Now, we are going to demonstrate that $S^*$ is the optimal sequence.
The rough idea is that the optimal solution of ILP is always implementable by ESD.
The reason is that an edge always probed earlier than one with higher order.
Otherwise, let lower-order edge be probed first can increase the value of the objective function.

Formally, it is stated as follows.

\begin{theorem}
	The output sequence $S^*$ of our algorithm is the sequence that maximizes the cardinality in the ESD algorithm.
\end{theorem}

\begin{proof}
	The proof of this theorem consists of two parts. Case (1): We show that any sequence of edges corresponds to a point in the ILP's solution space. This property indicates that the maximum value of the ILP is no smaller than the maximum expected number of matched edges.
	Case (2): We show that the sequence $S^*$ constructed by the mapping process corresponds to the optimal solution of the ILP.
	In other words, the expected matched edges of $S^*$ is the same as the maximum value of the objective function. It indicates that the maximum value of the ILP is no bigger than the maximum expected number of matched edges.
	
	Case (1): Given a sequence of edges $S=s_1,s_2,\ldots,s_m$, we can construct the variables in the ILP as follows.
	For any $i<j$ and $e_i$ and $e_j$ have a common edge. $\delta_{ij}=1$, otherwise $\delta_{ij}=0$.
	Let $o_i$ equal to $\sum_{j=1}^{m}\delta_{ji}$.
	$l_{ij}=1$ if and only if $o_i\geq j$.
	We call this process as \textit{reverse mapping process}.
	In this way, we have map $S$ to a point in the solution space of the ILP. 
	Further, the value of objective function under this construction is exactly the expected number of matched edges.
	So we conclude the maximum objective function value is greater than or equal to the maximum expected number of matched edges.
	
	Case (2): We are going to show that $S^*$ corresponds to the optimum solution of the ILP by reverse mapping. 
	To prove this, we only need to prove that the values of all the variables are the same for 1. the optimum ILP solution where $S^*$ is mapped from and 2. the variables generated by reverse mapping from $S^*$.
	Then, we show that we only need to compare the values of $\delta_{ij}$'s.
	\begin{itemize}
		\item In the optimal solution, according to Lemma \ref{lmm}, $l_{ij}$'s are decided by $o_i$'s.
		Further, $o_i$'s are decided by $\delta_{ij}$'s according to the equation \ref{oo}.
		So, when $\delta_{ij}$'s are fixed in  the optimal solution, the values of all the other variables are fixed.
		\item For the variables generated by reverse mapping from $S^*$,  $l_{ij}$'s, $o_i$'s and $\delta_{ij}$'s follow the same relationship as the former case. In details, $l_{ij}=1$ if and only if $o_i\geq j$, and $o_i=\sum_{j=1}^{m}\delta_{ji}$. So if the two cases have the same $\delta_{ij}$'s, all the variables are the same.
	\end{itemize}
	To prove that the two cases have the same $\delta_{ij}$'s, we need the following lemma.
	\begin{lemma}\label{lmmcore}
		In the optimum solution of ILP, if $o_j\leq o_i$, $\delta_{ij}=0$
	\end{lemma}
	\begin{proof}
		We prove this lemma by contradiction.
		By contradiction, we assume that in the optimum solution, there exist $i$ and $j$ satisfying $o_j\leq o_i$ and $\delta_{ij}=1$.
		By Lemma \ref{lmm} and Equations \ref{simple} to \ref{fobj}, in the optimum solution, the Equation \ref{simple} is equivalent to the objective function of the ILP.
		So, in the optimum solution, the value of the objective function is as follows.
		\begin{equation}
		obj_1=\sum_{i=1}^m p(1-p)^{o_i}
		\end{equation}
		Then, let $\delta_{ij}=0$ and $\delta_{ji}=1$ and keep the other $\delta$'s unchanged.
		We call the new values of $o_i$ and $o_j$ to be $o_i'$ and $o_j'$.
		We have $o_j'=o_j-1$ and $o_i'=o_i+1$.
		Now, we consider the point in the feasible space where $l_{ij}=1$ iff $o_i\geq j$.
		The value of the objective function at this point is as follows.
		\begin{eqnarray}
		obj_2&=& obj_1+p(1-p)^{o_j-1}+p(1-p)^{o_i+1}\nonumber\\&&-p(1-p)^{o_j}-p(1-p)^{o_i}
		\end{eqnarray} 
		Thus, we have:
		\begin{eqnarray}
		obj_2-obj_1&=& p(1-p)^{o_j-1}+p(1-p)^{o_i+1}\nonumber\\&&-p(1-p)^{o_j}-p(1-p)^{o_i}\nonumber\\
		&=& p(1-p)^{o_j-1}(1+(1-p)^{o_i-o_j+2}\nonumber\\&& -(1-p)-(1-p)^{o_i-o_j+1})\nonumber\\
		&>& 0\nonumber
		\end{eqnarray}
		So, we find a feasible point whose objective function's value is larger than optimum.
		A contradiction.
	\end{proof}
	Now, we consider a pair of edges $e_i$ and $e_j$. 
	\begin{itemize}
		\item If $\delta_{ij}=1$, by Lemma \ref{lmmcore}, $o_i<o_j$. 
		Then, in $S^*$, $e_i$ is ahead of $e_j$.
		Then, after reverse mapping, as $e_i$ and $e_j$ have a common vertex and $e_i$ is ahead of $e_j$, $\delta_{ij}=1$.
		\item If $e_i$ and $e_j$ are disjoint, after mapping and reverse mapping, $\delta_{ij}=0$
		\item If $\delta_{ij}=0$ but $e_i$ and $e_j$ have a common vertex, $e_j$ is ahead of $e_i$ in $S^*$.
		After reverse mapping, $\delta_{ij}$ still equals to 0.
	\end{itemize}
	So, all the $\delta_{ij}$'s are the same between the optimum solution where $S^*$ is generated from and the feasible point $S^*$ corresponds to in the ILP.
	Above all, we have finished the proof for the second case. 
\end{proof}

\subsection{Complexity analysis}

Even though solving ILP is an NP-Hard problem, instances at a small scale is solvable by many ILP solvers.
We care about the number of entries in our ILP.
By direct observation, we find the number of entries can be bounded by $O(|E|^2)$.
However, with carefully analysis, we reduce it to $O(|V||E|)$.

\begin{theorem}
	The ILP only has $O(|V||E|)$ entries.
\end{theorem}
\begin{proof}
	In the ILP, each $\delta_{ij}$, $o_i$ or $l_{ij}$  appears no more than 2 times.
	So, we only need to show that the number of variables is $O(|V||E|)$.
	\begin{itemize}
		\item The number of $o_i$'s is $|E|$.
		\item For each pair of neighboring edges $e_i$ and $e_j$, there are two corresponding variables $\delta_{ij}$ and $\delta_{ji}$. The total number of pairs of neighboring edges can be counted as follows. For an arbitrary edge $e_i$ if $e_j$ and $e_j$ has a common vertex. The common vertex has two alternatives and the other end of $e_j$ has $|E|-2$ alternatives. So, the number of $\delta$'s is $O(|V||E|)$.
		\item The number of $l_{ij}$'s is exactly the sum of $n_i$'s. Each pair of neighboring edges increases $\sum_{i=1}^m n_i$ by 2. So, the number of $l_{ij}$'s is $O(|V||E|)$.
	\end{itemize}
\end{proof}

\section{Extensions of our algorithm}
Our algorithm can perform well on small graphs.
However, for larger graphs, ILP is too slow.
As our algorithm is the fastest way to get the optimum sequence for serial dictatorship to the best of our knowledge, we need find methods to approximate the optimum sequence for larger graphs.

Further, ESD is a non-adaptive algorithm.
In other words, during the executing of ESD algorithm, the order stays unchanged without using considering the results of the previous probings.   
However, based on the ILP, we can also extend the algorithm to an adaptive algorithm.

\subsection{Force mapping}
When the size of the graph is big enough, the ILP can not be solved in an acceptable time limit.
However, our mapping process can always get a feasible solution.

The rough idea of the force mapping is as follows.
We set a maximum acceptable time $T$.
Terminate the ILP at time $T$.
At that time, the \textit{variable profile} (the values of all the variables) is denoted by $P$ and the corresponding objective value is denoted by $obj$.
Then, map $P$ to a sequence $S$ by the mapping process.
$S$ is used as the sequence for ESD algorithm.

\subsection{Improvements on optimum serial dictatorship}
Actually, the payoff of Algorithm \ref{alg:full} can be further improved.
However, such improvements breaks some great properties ESD holds, such as offline executable and deterministic.
At the same time, these improvements induce more computation.

\subsubsection{Adaptive ESD}
To improve Algorithm \ref{alg:full} on payoff, we can use adaptive ESD.
The rough idea is that only use Algorithm \ref{alg:full} to decide the next $k$ probed edges, and repeat this process. It is formally stated in Algorithm \ref{alg:adapt}.

\begin{algorithm}
	\caption{Optimum serial dictatorship}
	\label{alg:adapt}
	\begin{algorithmic}[1]
		\Require
		A graph $G=(V,E)$ and a positive integer $k$.
		\Ensure A set of disjoint edges $A$, such that every edge in $A$ exists.
		\State $A=\emptyset$
		\While{$G$ is non-empty}
		\State Run the ILP shown in Equations \ref{stobj}-\ref{oo} on graph $G$. The variable profile for the optimum solution is denoted by $P$. 
		\State Map $P$ to a sequence $S=s_1,s_2,\ldots$ by mapping process. 
		\For{$i=1,2,\ldots,min(k,length(S))$}
		\If{Both ends of $s_i$ are still in $G$}
		\State Prob $s_i$ to match
		\If {$s_i$ is proved to be exist.}
		\State Remove the two ends of $s_i$ and all edges attached to them from $G$.
		\State Add $s_i$ to $A$
		\Else
		\State Remove $s_i$ from $G$
		\EndIf
		\EndIf
		\EndFor
		\EndWhile
		\State Output $A$
	\end{algorithmic}
\end{algorithm}

When $k$ is small, the algorithm \ref{alg:adapt} tends to output a better solution but costs more time to compute. 
\begin{theorem}
	For any positive integer $k$, the payoff of Algorithm \ref{alg:adapt} is same as or better than the payoff of Algorithm \ref{alg:full}
\end{theorem}
The rough idea of the proof is as follows. We prove this by induction. 

The drawback of \ref{alg:adapt} is that it needs interaction with computer.
While Algorithm \ref{alg:full} can be totally offline with a more clear scheme.
\subsubsection{Skip ratio}
\cite{bansal2012lp} gets some bounds on weighted stochastic matching.
A technique they used to get bounds is adding a skip rate when matching following a sequence.

we can also add a skip ratio $r (0\leq r<1)$ to algorithm \ref{alg:full}.
When running the ESD algorithm, each time, we skip each edge with probability $r$.
In this way, it is possible to get a higher payoff.

If we directly add $r$ to our ILP, it will incurs great difficulty for solving it.
So, we only add the $r$ on the optimum sequence.
Then the value of the payoff on a sequence $S$ is as follows.
\begin{equation}
obj(s)=\sum_{i=1}^{m}(1-r)p(1-(1-r)p)^{\sum_{j=1}^{i}\delta(s_j,s_j)}
\end{equation} 

In practice, this methods only have little effect on the payoff.
Further, this method is not deterministic.

\section{Experiments}

our experimental results consists of two parts.
In the first part, we show the running time performance of Algorithm \ref{alg:full}.
Our results show that our algorithm outperforms naive enumeration significantly and can scale up to the case where $|V||E|=200$.
In the second part, we compare the revenue with other methods.
As the running time of all the algorithms are not much relied on the acceptance ratio (we confirm this by some experiments), we set all the acceptance ratio as 0.7.

\subsection{Scalability}

As there is no previous literature on optimum serial dictatorship in stochastic matching, we compare our algorithm to the most naive algorithm, which enumerates all the permutations of the edges and find the best of them.
\begin{table}
	\caption{Running time comparison between Algorithm \ref{alg:full} and the naive algorithm.}
	\label{tab1}
	\begin{tabular}{ccccc}
		\hline
		n,m,algorithm & maximum & minimum & median & mean\\ \hline
		5,6,ILP & 0.085& 0.043& 0.054& 0.056\\
		5,6,naive & 0.007& 0.006& 0.007& 0.007\\
		6,8,ILP & 0.131& 0.062& 0.099& 0.099\\
		6,8,naive & 0.587& 0.565& 0.575& 0.575\\
		6,9,ILP & 0.245& 0.078& 0.128& 0.130\\
		6,9,naive & 6.467& 6.271& 6.353& 6.349\\ \hline
	\end{tabular}
\end{table}

Table \ref{tab1} shows the running time of Algorithm \ref{alg:full} and the naive algorithm for three cases.
For each case, we test 20 instances with $n$ as the number of vertices and $m$ as the number of edges.
``ILP'' and ``naive'' stands for the two algorithms.
The unit of all the numbers in the table is seconds.
As the running time of the naive algorithm is $\Theta(|E|!\times |E|^2)$.
It can be concluded that when the number of edges is greater than 7, Algorithm is significantly faster.

\begin{table}
	\caption{Running time of Algorithm \ref{alg:full} for larger instances.}
	\label{tab2}
	\begin{tabular}{ccccc}
		\hline
		n,m & maximum & minimum & median & mean\\ \hline
		10,14 & 0.781&0.157&0.234&0.314\\
		10,16 & 9.510&0.281&0.664&1.864\\
		10,18 & 19.157&2.337&3.447&5.162\\
		10,20 & 65.872&13.830&17.763&22.424\\
		10,21 & 523.552&25.045&62.312&101.141\\ \hline
	\end{tabular}
\end{table}

Table \ref{tab2} shows the performance of Algorithm \ref{alg:full} on larger cases.
We find that the worst case running time of our algorithm increase dramatically.
We set $n=10$, because only under this condition, the graphs with $|V||E|\approx200$ is neither too sparse nor too dense.
Under such condition, our algorithm can often solve instances with $m=20$ within 1 minutes.
Although the scalability of our algorithm is far from perfect., it has been much better than enumeration with the help of a highly non-trivial ILP.

\section{Conclusion}
In the field of stochastic matching, previous study concentrates on worst case bounds, rather than ``optimum''.
We move forward the first step towards ``optimum'' by a talent ILP construction for the optimum serial dictatorship.
Our algorithm performs much better than enumeration.
We show that serial dictatorship can yield very good payoff and can be extended in many ways.
Serial Dictatorship has many good properties that make it practical.
Our algorithm can be applied in real world directly, showing the power of artificial intelligence.
The most important point is that our study shows the possibility of exploring the ``uncertain''` matching world.


\newpage
\bibliographystyle{named}
\bibliography{matching}

\end{document}


